\documentclass{article}

\usepackage[english]{babel}
\usepackage{hyperref}
\usepackage{xspace}

\usepackage{kappa}

\newcommand{\kpp}{Kappa\xspace}
\newcommand{\lms}{LMS\xspace}
\newcommand{\lmsk}{\lms-\kpp}

\title{\lmsk: \\ User manual}
\author{Ricardo Honorato-Zimmer and Sandro Stucki}

% Agents
%
% \newagent[rectangle, rounded corners=1em, minimum width=10ex,%
%   minimum height=4em, text width=1cm, text depth=1.3em,%
%   align=center]{\rnap}{RNA pol}{n1//west/n1,n2//south/n2,n3//east/n3}
% \newagent{\sig}{$\sigma$}{n//south/n}
\newagent[rectangle, rounded corners=1em, align=center,%
  minimum width=16ex, minimum height=3em]{\rib}{Ribosome}{%
    n1//west/n1,n2//south/n2,n3//east/n3,a//north/a}
\newagent{\nuc}{R}{b//north/b,u//west/u,d//east/d,t//south/t}
\newagent{\aac}{aa}{b//south/b,u//west/u,d//east/d,t//north/t}

\begin{document}

\maketitle

\section{Introduction}
\lmsk is an \emph{embedded domain-specific language} for the
\kpp language (\url{http://www.kappalanguage.org/}) on Scala
(\url{http://www.scala-lang.org/}).  That is, it enables you
(the user) to write \kpp models as objects within Scala, making
it possible to write models programmatically.

In this manual, we first introduce how to write \kpp patterns in
Scala.  \kpp patterns are used to define the rules and initial state
of a model among other uses.  Second, we introduce \emph{contact graphs}.
Once we have defined the contact graph, the rules and the initial
state of a model, we can run it.  However, without first stating what
patterns we want to track during the simulation, running a simulation
is meaningless.  For this we define observables that are patterns
which count the number of instances of a pattern in the mixture.
To visualise a simulation we may plot these observables.

Later, we will dive into the different extensions that have been
developed on top of \kpp.  These extensions allow us to describe aspects
of biological or biochemical systems that are hard or impossible to
express inside \kpp itself.


\section{Patterns, Mixtures and Rules}
In \kpp, an agent has a set of \emph{sites} called the agent's
interface.  Each of these sites can either be bound to another site
(of any agent) or be free.  Also, each site can have a state, commonly
represented by a name (but not necessarily so as we will see in some
of the extensions).  In addition to a set of sites, each agent has a
\emph{type}.  Agents commonly represent proteins while sites represent
domains or motifs.  However, this can be otherwise depending on the
modelling problem.  A \kpp pattern is a collection of such agents
whose sites might be bound together.  Complementary to the definition
of a pattern is the definition of an \emph{embedding}: a pattern can
be embedded into another if the former can be included (superposed)
into the latter and the inclusion respects the connectivity and the
state of sites.  To illustrate these two complementary ideas, let's
take as an example the binding of the ribosome to an mRNA chain
(Fig.~\ref{fig:patterns}).
\begin{figure}[t]
  \begin{center}
    \resizebox{\linewidth}{!}{%
      \begin{tikzpicture}
        \rib{rib}{(0pt,0pt)}{n1,n2,n3};
        \nuc{n1}{(-60pt,-60pt)}{b,d};
        \nuc{n2}{(0pt,-60pt)}{b,d,u};
        \nuc{n3}{(60pt,-60pt)}{b,u};

        \draw[link, rounded corners=5pt] (rib-n1) -| (n1-b);
        \draw[link, rounded corners=5pt] (rib-n3) -| (n3-b);
        \draw[link] (rib-n2) -- (n2-b);
        \draw[link] (n1-d) -- (n2-u);
        \draw[link] (n2-d) -- (n3-u);

        \draw[very thick,->] (n2.south) ++(0pt,-15pt) -- ++(0pt,-30pt);

        \begin{scope}[shift={(0pt,-240pt)}]
          \rib[fill={orange!40}]{rib}{(0pt,0pt)}{%
            n1/fill=orange!20/west/n1,n2/fill=orange!20/south/n2,%
            n3/fill=orange!20/east/n3,a};
          \nuc[fill=orange!40]{n1}{(-60pt,-60pt)}{%
            b/fill=orange!20/north/b,d/fill=orange!20/east/d,u,t};
          \nuc[fill=orange!40]{n2}{(0pt,-60pt)}{%
            b/fill=orange!20/north/b,d/fill=orange!20/east/d,%
            u/fill=orange!20/west/u,t};
          \nuc[fill=orange!40]{n3}{(60pt,-60pt)}{%
            b/fill=orange!20/north/b,d,u/fill=orange!20/west/u,t};

          \nuc{n4}{(-120pt,-60pt)}{b,d,u,t};
          \nuc{n5}{(-180pt,-60pt)}{b,d,u,t};
          \nuc{n6}{(120pt,-60pt)}{b,d,u,t};
          \nuc{n7}{(180pt,-60pt)}{b,d,u,t};
          % \sig{s}{(-120pt,-11pt)}{n};

          \aac{aa1}{(0pt,60pt)}{b,u,t};
          \aac{aa2}{(-60pt,60pt)}{b,u,d,t};
          \aac{aa3}{(-120pt,60pt)}{b,u,d,t};
          \aac{aa4}{(-180pt,60pt)}{b,d,t};

          \draw[link, rounded corners=5pt] (rib-n1) -| (n1-b);
          \draw[link, rounded corners=5pt] (rib-n3) -| (n3-b);
          \draw[link] (rib-n2) -- (n2-b);
          \draw[link] (n1-d) -- (n2-u);
          \draw[link] (n2-d) -- (n3-u);
          \draw[link] (n5-d) -- (n4-u);
          \draw[link] (n4-d) -- (n1-u);
          \draw[link] (n3-d) -- (n6-u);
          \draw[link] (n6-d) -- (n7-u);
          % \draw[link] (n4-b) -- (s-n);

          \draw[link] (rib-a) -- (aa1-b);
          \draw[link] (aa1-u) -- (aa2-d);
          \draw[link] (aa2-u) -- (aa3-d);
          \draw[link] (aa3-u) -- (aa4-d);

          \state{n1}{t}{A};
          \state{n2}{t}{C};
          \state{n3}{t}{C};
          \state{n4}{t}{G};
          \state{n5}{t}{U};
          \state{n6}{t}{U};
          \state{n7}{t}{G};

          \state{aa1}{t}{T};
          \state{aa2}{t}{L};
          \state{aa3}{t}{P};
          \state{aa4}{t}{N};
        \end{scope}
      \end{tikzpicture}
    }
  \end{center}
  \caption{Kappa patterns and embedding.}
  \label{fig:patterns}
\end{figure}
In the figure we see two \kpp patterns, one above and one below the
arrow.  The arrow represents an embedding, that is, it tells us that
the pattern above can be included in the pattern below.  To make this
more apparent, we have highlighted the agents and sites in the second
pattern where the inclusion can be made.  Of course, nothing forbids
the second pattern to have two or more inclusion places in which case
we would have more than one embedding for the pattern above.  This is
crucial when we talk about the reaction mixture, where normally the
left-hand side of each rule (which itself is a pattern) will have many
embeddings into the reaction mixture and each of those embeddings will
be one possible way in which we can fire the rule.

\section{Contact graphs}

\section{Observables and Plotting}

\section{First example: ...}

\section{Perturbations} % and some other advanced things like ...?

\section{Extensions}
\subsection{GeEK: Geometrically enhanced Kappa}

\subsection{Thermodynamic Kappa}

\subsection{Spatial Kappa}

\subsection{Hierarchical Kappa} % or a better name?

\section{Kappa BioBrick Framework}

\section{Fragmentation and ODEs}

\section{Performance} % comparisons with KaSim and NFSim?

\end{document}
